\documentclass{standalone}
% preamble: usepackage, etc.
\begin{document}

\thesischapterexordium

\section{研究工作的背景与意义}
随着城市交通网络的发展和交通工具的进步,针对路径规划问题设计的各种系统和算法逐渐在现实生活中被广泛的应用,比如在日常驾驶出行或乘坐公共交通等情景中,人们越来越频繁的使用这一功能。同时随着交通网络复杂度提高,规模增大,多种交通工具的出现以及城市堵车给城市交通网络带来的随机性和动态性,这给传统的路径规划问题带来了全面的挑战。同样,随着自动驾驶技术的发展,路径规划问题作为自动驾驶的导航模块中重要的部分同样扮演者不可或缺的角色。因此如何更好的设计出更强大,更具灵活性的,更高效的路径规划算法具有重要的发展意义,而这也成为了当前各国学者的研究热点之一。

% \footnote{脚注序号“\ding{172},……,\ding{180}”的字体是“正文”,不是“上标”,序号与脚注内容文字之间空1个半角字符,脚注的段落格式为:单倍行距,段前空0磅,段后空0磅,悬挂缩进1.5字符;中文用宋体,字号为小五号,英文和数字用Times New Roman字体,字号为9磅;中英文混排时,所有标点符号(例如逗号“,”、括号“()”等)一律使用中文输入状态下的标点符号,但小数点采用英文状态下的样式“.”。}

\section{路径规划问题的国内外研究历史与现状}
路径规划问题,最早被转换为图论中的最短路径问题,最短路径问题根据图的类型不同,分为无向图上的最短路径、有向图上的最短路径、单源最短路,多源最短路等等。Edsger W. Dijkstra 最早于1956年,提出了著名的 $Dijkstra$ 算法,给出了单源最短路算法,随后包括 $Bellman-Ford$, $A^* search$, $Floyd-Warshall$ $Johnson's algorithm$, $Viterbi algorithm$等被相继提出,在一定程度上都降低了最短路算法的时空复杂度。\par
但随着交通环境复杂度的增加(大陆级别的地图范围)和用户需求的增加(例如需要考虑多种交通工具的组合,倾向选择公共交通方式,选择非收费路段等),先前提出的基于静态网络和单一的度量方法都已经无法适应。因此各国学者和研究机构,例如Microsoft,Karlsruhe Institute of Technology等针对不同的场景,提出了改进的算法。

\section{本文的主要贡献与创新}
我们的主要贡献包括:1)首次将端到端的强化学习应用到了路径规划算法,并证明了其可用性。2)使用强化学习解决了传统算法难以解决的某些路径规划场景。

\section{本论文的结构安排}
本文的章节结构安排如下:\par
论文分为五个部分,在第一章,我们给出了路径规划问题的基本数学形式,同时我们给出了几种基于基本路径规划问题的变形,例如带有额外转弯惩罚的路径规划问题,带有行驶长度限制的电动车路径规划问题,以及动态地图环境下的路径规划。这些特殊的场景是传统的最短路径算法所难以解决的,我们将通过强化学习的方法很好的解决这些场景;在第二章,我们先给出传统的基于搜索的路径规划算法,例如 $A^{*}$,$Dijkstra$ 等,并给出为了解决以上特殊场景而做出的一些改进算法。同时我们试着简单分析这类方法的缺点,例如不够灵活,无法应对复杂动态环境或更复杂的用户需求等;在第三章,我们给出强化学习的基本概念和相关方法,包括了强化学习的基于智能体建模的方法和基于马尔科夫决策过程的控制结构。以及强化学习中的基本概念,如价值函数,策略函数等,最后着重介绍为了解决路径规划问题使用的基于值函数的方法;在第四章,我们将介绍我们的方法,首先将路径规划问题表述为一个标准的马尔科夫决策过程,然后针对不同的环境,我们设计了基于 $Q-learning$ 的强化学习算法。在第五章,我们给出了具体的代码实现方法和实验结果;最后给出全文的总结和未来的研究方向。

\end{document}