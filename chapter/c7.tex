\documentclass{standalone}
% preamble: usepackage, etc.
\begin{document}

\chapter{全文总结与展望}
\section{全文总结}
在该文中,我们设计并实现了基于强化学习的路径规划算法。我们首先引入了路径规划问题的定义和相关扩展,然后介绍了基于搜索的传统方法,并简单分析了其局限性。随后我们介绍了强化学习的相关背景,例如马尔科夫决策过程,智能体建模方法,最后详细给出了强化学习的 Q-learning 算法。在五章,我们给出了我们的算法,包括如何将路径规划问题转化为一个标准的强化学习环境,然后对要解决的三个场景分别提出了基于 Q-learning 的算法,给出了相关的定义和算法流程。在第六章,我们给出了实验的实现细节,包括了代码设计的核心思路和用到的设计模式。然后给出实验参数设置和实验结果,通过实验结果可以看出我们的算法很好的解决了相应的场景,
\section{后续工作展望}
在我们的工作中,我们的算法解决了小规模地图下的多个场景下的路径规划问题,但在现实路径规划系统中,问题的规模远远大于我们的实验设计,因此如何将我们的算法迁移到大规模场景下是未来的研究方向之一,这首先将影响到我们算法的设计,基于查表法的 Q-leanring 难以处理如此大规模的数据,因此可以在接下来的研究中通过基于函数近似的方法,利用神经网络等方法,例如 DQN 等近些年提出的深度强化学习算法解决该问题。结合强大的计算能力和大规模的神经网络,这类方法能够处理更加复杂的路径规划问题\par
第二的方向为,如何设计一个基于强化学习的通用的路径规划算法,在我们的工作中,虽然解决了部分场景,算法并不属于同一个框架,在部分设计上存在不同的地方,虽然相比传统方法已经具有很高的灵活性,但也在一定程度上限制了强化学习方法的进一步应用,因此在未来的研究中,可以关注如何设计更加灵活的状态,行为和奖赏函数以使得模型的泛化能力提高。\par
第三,如何设计一个离线算法也是未来的研究方向之一。我们的模型现在只能给出在线的规划算法,但常见的路径规划系统往往具有离线规划的能力,即单次询问即可给出完整的规划结果,所以如果把强化学习实际应用到现实系统中,如何实现离线规划也是重要的研究方向。\par
最后,在基于环境交互进行学习的同时,如何利用现有的行驶数据进行学习也是未来的重要的研究方向,这也不仅仅是该问题上的发展方向,也是强化学习该领域的一个重要方向。如果能很好的利用现实世界采集的真实数据,会大大降低模拟环境的误差带来的问题,提升模型的效果。
\end{document}