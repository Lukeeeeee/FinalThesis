\documentclass{standalone}
% preamble: usepackage, etc.
\begin{document}
	
\begin{chineseabstract}

在一个交通网络中,给定起点和终点,给出一条最优路径已经作为一种算法,广泛的应用到了实际生活中。这类算法一般被称为路径规划算法,或者最短路径算法。随着交通网络的复杂,多种交通工具的发展,传统的基于静态地图的路径规划算法已无法适应日渐复杂的环境,如何实现出更加强大和灵活的路径规划算法已经成为各国学者的一个研究重点。

随着人工智能和机器学习的快速发展,强化学习,作为一个该领域的一个重要分支,展现它在很多领域的应用潜力,比如围棋等。作为一个决策和控制的框架,强化学习同样可以被应用在路径规划系统。在我们的工作中,我们使用强化学习解决路径规划问题,并且解决了多个不同的路径规划场景问题,特别是解决了一些传统的基于搜索的路径规划算法,比如 $A^{*}$,$Dijkstra$ 等不能解决的场景。



\chinesekeyword{强化学习,路径规划, $Q-learning$, $Dijkstra$, $A^*$, 转弯惩罚,电动车路径规划,动态环境}
\end{chineseabstract}

\end{document}