\documentclass{standalone}
% preamble: usepackage, etc.
\begin{document}
	
\begin{chineseabstract}

在一个交通网络中,给定起点和终点,给出一条最优路径已经作为一种算法,广泛的应用到了实际生活中。这类算法一般被称为路径规划算法,或者最短路径算法。随着交通网络的复杂,多种交通工具的发展,传统的基于静态地图的路径规划算法已无法适应日渐复杂的环境,如何实现出更加强大和灵活的路径规划算法已经成为各国学者的一个研究重点。

随着人工智能和机器学习的快速发展,强化学习,作为一个该领域的一个重要分支,展现它在很多领域的应用潜力,比如围棋等。作为一个决策和控制的框架,强化学习同样可以被应用在路径规划系统。在我们的工作中,我们使用强化学习解决路径规划问题,并且解决了多个不同的路径规划场景问题,特别是解决了一些传统的基于搜索的路径规划算法,比如 $A^{*}$,$Dijkstra$ 等不能解决的场景。

论文分为五个部分,在第一章,我们给出了路径规划问题的基本数学形式,同时我们给出了几种基于基本路径规划问题的变形,例如带有额外转弯惩罚的路径规划问题,带有行驶长度限制的电动车路径规划问题,以及动态地图环境下的路径规划。这些特殊的场景是传统的最短路径算法所难以解决的,我们将通过强化学习的方法很好的解决这些场景;在第二章,我们先给出传统的基于搜索的路径规划算法,例如 $A^*,Dijkstra$等,并给出为了解决以上特殊场景而做出的一些改进算法。同时我们试着简单分析这类方法的缺点,例如不够灵活,无法应对复杂动态环境或更复杂的用户需求等;在第三章,我们给出强化学习的基本概念和相关方法,包括了强化学习的基于智能体建模的方法和基于马尔科夫决策过程的控制结构。以及强化学习中的基本概念,如价值函数,策略函数等,最后着重介绍为了解决路径规划问题使用的基于值函数的方法;在第四章,我们将介绍我们的方法,首先将路径规划问题表述为一个标准的马尔科夫决策过程,然后针对不同的环境,我们设计了基于 $Q-learning$ 的强化学习算法。在第五章,我们给出了具体的代码实现方法和实验结果;最后给出全文的总结和未来的研究方向。

\chinesekeyword{强化学习,路径规划}
\end{chineseabstract}

\end{document}