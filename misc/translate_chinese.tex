\documentclass{standalone}
% preamble: usepackage, etc.
\begin{document}
\thesistranslationchinese

\section*{Q-路由算法和最短路径路由规划算法算法比较}
作者:F. Tekiner, Z. Ghassemlooy and T.R. Srikanth\par
下载链接:http://bit.ly/2LsIlNo
\subsection*{摘要}
在这篇论文中,我们将自适应 Q 学习方法,双重强化学习 Q 路径规划算法与常见的路由算法进行了比较。Q路径规划算法在每一个节点嵌入了一个学习策略,使得通过自身的调整产生一个同步的路径规划信息,从而达到最快到达时间。不同于 Q 路径规划方法,最短路径算法基于节点连接上的最晚到达时间进行规划,而无视节点上的网络交通状况。在这篇论文,我们加入了带有故障的情况下的模拟环境。实现结果显示出自适应算法比传统的非自适应算法在一个有缺陷的环境下具有更好的表现。
\subsection*{简介}
在当前快速发展的互联网,网络状况时常变化,并且网络部分节点经常以不可预知的方式发生种种错误而无法连接。因此,我们需要一个管理网络状况的算法,使得从源点到目标点的网络包能够以一个合理的时间被送达。\par
将网络包从源节点传输到目标节点的过程在一个网络中被称为路由。通常,包可以在中间点上作多次跳转使得其到达目标点。在每一个节点,当包被收到后,节点进行存储,并将其跳转到下一个节点,直到包到达目标点。\par
路由算法根据其是否能够适应网络交通模式而被分为动态的或适应性的,以及静态的或非适应性的算法。在非适应性的算法中,路由决策依赖于局部的信息,因此路由问题分布式的存在于网络中。另外,在非适应性算法中, 其需要全局的知识,而这使得其更难以应对网络中的变化。\par
最短路径路由算法是一个被广泛应用和部署的非自适应的算法,该算法基于著名的 Dijkstra 最短路径规划算法。在这个算法中,到达一个节点的包采用相同的最短路径到达目标点。这类方法的最大缺点在于,它没有动态的考虑网络状况的变化。例如,由于中间节点的拥堵而导致包在到达目标点值钱被延迟。在这样的情况下,将包通过其他的非最短路径可能会使得到达时间更早。\par
现有的路由算法缺少智能性,它们需要人的参与和协助以保证它们能够应对网络的变化和故障。然而引入智能,并不一定提高算法的性能,但提高了算法对网络变化的适应能力。\par
在近些年,基于智能体的系统和强化学习算法被广泛应用在路由算法中。这是由于这类方法不需要任何监督过程,并且具有分布式的性质。群体智能体,特别是基于蚁群的系统,以及Q-Learning,基于距离向量的混合智能体路径规划算法,展现了其优越的结果。\par
因此在本文中,我们重点关注强化学习中的Q-Learning 算法。Q-路由算法是一个适应性的路由算法,它通过学习邻居的路由信息进行路由决策。初始时,算法建立一个基于送达时间的路由表,即 Q 值表,这些送达时间表在每次该节点传输一个包到目标点后进行更新。基于此信息,当网络发生堵塞时,一个节点可以选择替代的路径而避开堵塞节点,从而使得包的传输时间降低。在这种学习模式中,路由信息在网路中的全部节点中进行同步。
\end{document}